\chapter{Introduction}
\label{chp:introduction}
After the introduction of the Internet on 60's and world wide web on 90's, there are more and more people connected each other using this technology. Initially, the architecture used to build the Internet is Peer-to-peer network although the most popular one is \textit{client and server}. A \textit{server} holds the content and delivers it to a user at the \textit{client} side, thus burden the server. Currently, recent research shows that peer-to-peer Internet already back at its prime by dominating the traffic\cite{2015:internettraffic:sandvine}.

Among all the peer-to-peer usage in the Internet, file-sharing is the most popular one. It started with Napster in 1999 to share music file between its users. It shut down at 2001 and immediately followed by Kazaa and Gnutella afterward. Both services allowed the user to share not only music file but also another type of files. Currently, both already shut down because of legal and performance issues. In Gnutella case, majority of users (70\%) stopped to share their files. Moreover, about half of the communication only served by top 1\% of the community \cite{2000:freeridegnutella:adar}. Gnutella suffers from a social phenomenon called \textit{freeriding} on majority of its users.

Freerider, can cause several problems, especially in peer-to-peer network. First and foremost, freeriding behaviour can lead to vulnerabilities in the system. With only few of the user provide the service for many, it is eventually become more centralized than decentralized system. Another well known problem caused by freeriders is the degradation of system performance \cite{2000:freeridegnutella:adar}. If freeriders become majority in file-sharing peer-to-peer system, as they occupy significant amount of resource, eventually bottleneck in the system will occur. As the time goes, honest peer may not feel satisfied and decided to leave the system. With important peer leave the system, it will degrade more and lost the file that used to be served by leaving peer. The system become unhealthy and sooner or later will be completely left by its peers.

\section{BitTorrent protocol}
\bt~\cite{2003:bittorrent:cohen}, nowadays, stand as \textit{de facto} file-sharing protocol on top of peer-to-peer network. It survives until now because \bt~ is a \textit{protocol} that can be implemented by anyone, instead of service that Napster, Kazaa, and Gnutella used to have. Static \texttt{.torrent} file, which contains information such as tracker addresses and unique hash value of this swarm, is created by peer who wants to publish their files. \textit{Tracker} is responsible for monitors the distribution and progress of file and peers in a swarm. Peer uses information in \texttt{.torrent} file to connect each other.

\bt~uses \textit{tit-for-tat} mechanism to reward good behavior and punish bad behavior. This mechanism tried to solve fairness issue introduced by freeriding behavior \cite{2003:bittorrent:cohen}. \textit{Tit-for-tat} in \bt~ encourage user to only upload file to one who also has uploaded his file somewhere else. Freerider always getting low priority in this mechanism. In this way, \textit{tit-for-tat} incentivizes for user to upload a file. \bt protocol and its \textit{tit-for-tat} become a standard in file-sharing peer-to-peer system with many clients implemented this protocol.

\subsection{Tribler}
Tribler\footnote{\url{https://www.tribler.org/}} is peer-to-peer file sharing application developed at Delft University of Technology that compatible with \bt~protocol \cite{2008:tribler:pouwelse}. Tribler focused on security, fully decentralized system, and anonymity. Starts with ABC (Another \bt~Client), Tribler currently provides content discovery, channels concept, and reputation management in fully distributed manner. With Tribler downloaded from the official repository on the latest stable release (6.5.2) reaching  78440\footnote{\url{http://www.somsubhra.com/github-release-stats/ ?username=tribler&repository=tribler} (Accessed 3 September 2016)} times, it is desired to observe the usage of CMS with an adequate user base. We believe our work will be able to increase the overall swarm throughput by donating unused bandwidth on peer upstream.

All of the Tribler main components such as end-to-end encryption, channel discovery, and many others relied in database and dissemination system called \texttt{Dispersy} \cite{2013:dispersy:zeilemaker}. Dispersy maintain and perform the communication between Tribler peers in fully decentralized manner. Dispersy able to circulate the message in one-to-one or one-to-many within a group of node called \texttt{community}. User can adapt and implement its desired \textit{community} by itself. It is including how, what, and where the communication will occur.

Tribler implements several Dispersy \textit{communities} on its core function. \citeauthor{2016:tribler-techdebt:vos} summarize the recent community in Tribler. Important features such as channel discovery, search within community, end-to-end Tor-like operations, and currency mechanism shown in table \ref{tbl:community}. \textit{Channel} is a collection of torrent that has extra capabilities such as vote system, spam prevention, and comment (social) attributes. Every user can create his own channel, add and remove torrent to it, and maintain its activity. Worth to mention that Tribler implemented its own reputation system to incentivize user. Reputable user will get boost from other Tribler user, so it is beneficial in its own way.

In Tribler, there were several attemps to tackle freerider issue. Give-to-Get \cite{2008:givetogetvod:Mol} is one approach in peer-to-peer streaming video system. It works by give freerider only idle bandwidth slots and therefore their download speed will much slower\footnote{\url{https://www.tribler.org/Give-To-Get/}(Accessed 22 September 2016)}. There was also reputation management implemented in \textit{BarterCast4 community}, specifically to prevent freeriding in Tribler \cite{2009:bartercast:meulpolder}. It was used to spread the statistics about upload and download rate of a particular user \cite{2016:tribler-techdebt:vos}. And lastly, there is Multichain \cite{2015:multichain:norberhuis}, the anonymous tamper-proof interaction history that works on onion routing in Tribler network. Multichain projected to replace \textit{BarterCast4} as reputation system. Multichain has several improvement compared to \textit{BarterCast4} in aspect of security and accountability.

\begin{table}[tbp]
	\centering
	\caption{Overview of implemented Dispersy community in Tribler \cite{2016:tribler-techdebt:vos}.}
	\label{tbl:community}
	\begin{tabular}{|l|p{11cm}|}
		\hline
		\rowcolor[HTML]{EFEFEF} 
		\multicolumn{1}{|c|}{\cellcolor[HTML]{EFEFEF}{\color[HTML]{333333} \textbf{Community Name}}} & \multicolumn{1}{c|}{\cellcolor[HTML]{EFEFEF}{\color[HTML]{333333} \textbf{Purpose}}}                                                                                                                                     \\ \hline
		\textit{AllChannel}                                                                          & Used to discover new channels and to perform remote channel search operations.                                                                                                                                           \\ \hline
		\textit{BarterCast4}                                                                         & While currently disabled, this community was used to spread statistics about the upload and download rates of peers inside the network and has originally been created as a mechanism to prevent free-riding in Tribler. \\ \hline
		\textit{Channel}                                                                             & This community represents a single channel and is responsible for managing torrents and playlists inside that channel.                                                                                                   \\ \hline
		\textit{Multichain}                                                                          & This community utilizes the blockchain technology and can be regarded as the accounting mechanism that keeps track of shared and used bandwidth.                                                                         \\ \hline
		\textit{Search}                                                                              & This community contains functionalities to perform remote keyword searches for torrents and torrent collecting operations.                                                                                               \\ \hline
		\textit{(Hidden)Tunnel}                                                                      & This community contains the implementation of the Tor-like protocol that enables anonymity when downloading content and contains the foundations of the hidden seeder services protocol, used for anonymous seeding.     \\ \hline
	\end{tabular}
\end{table}

\section{Rewarding user contribution}
Freeriding behavior can be prevented by proper incentive mechanism. By showing goodness, specifically, by uploading data to others, user should get a reward. However, users are typically selfish and always tries to maximize their own benefit \cite{2015:incentivep2pgame:kang}. With unclear and non-obvious incentive mechanism, some peers who download a lot may or may not know that freeriding behavior is causing trouble for the system. Therefore, they could suffer from punishment. Reward and punishment can be in many forms such as right to download specific content, get higher download speed, and social acknowledgement.

To gain the reward is not as trivial as it sounds. Reward comes with good behavior which can be done by uploading content. This requires another user to actually download the content. In a community where the punishment is significantly severe, users typically very selective of its download activity. By downloading more, a user can be suspected with bad behavior that lead to punishment. Similar situation applies if the reward is insufficient. The user who want to get reward may need to standby for a long time waiting someone to download their files \cite{2013:survivepriv:jia}. This approach is inefficient, bandwidth wasting, but commonly practicable\cite{2013:survivepriv:jia}.

The focus on this thesis is to introduce credit mining system, a system to automatically upload prospected files. This system tries to find a collection of files which give relatively high reward if it uploaded in the future. As the system is implemented to increase user experience, it is implemented in such a way that it will not disturb any kind of user activities. Balancing between gaining high reward, consumed resource, and freeriding prevention become a key question of this thesis work.

\section{Document Structure}
This thesis is structured as follows. Chapter 2 discusses problem we intend to solve and related work of it. Chapter 3 presents the design of credit mining system integrated with Tribler. Implementation of the mechanism and it's experiment will be elaborated in chapter 4. Chapter 5 shows performance of credit mining system. At the end, chapter 6 concludes the work mentioning possible future work.


