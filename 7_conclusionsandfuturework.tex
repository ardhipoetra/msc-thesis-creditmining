\chapter{Conclusions and Future Work}
\label{chp:conclusionsandfuturework}

\section{Conclusions}
This thesis aims to solve the \bt~supply and demand misalignment introduced by freeriders. We devised an automatic mechanism to effectively gain credit in the existing credit system. We showed that it benefits both individuals and communities. The user can gain the credit without the need to seed for a long time. The swarms in the community can also get higher download speeds and content availability.

We then proposed the enhancement of \textit{credit mining system}, an autonomous system to download pieces from selected swarms in order to gain high upload ratio in the future. It can run on top of the traditional credit system that is already widely implemented in both private and public communities. The credit mining system finds swarms with high return potential, picks the rarest pieces, and uploads those to the community using all the available bandwidth.

We focused on the investment algorithm which is the core of credit mining. Two stages of the algorithm were presented. Both stages are important. The \textit{prospecting} stage filters a huge number of swarm while maintaining its resilience by not being solely dependent on a centralized tracker. The \textit{mining} stage only selects the best swarms, those that have a high gain potential. If necessary, it also stops problematic or low potential swarms. We also proposed the \textit{scoring} policy, a highly customizable method to quantify swarms into a score that can be compared with each other, which reduces any possible identical result.

The credit mining system is now fully integrated with Tribler. When enabled, it is tailored to not interfere with users' activity while downloading. Provided with an accessible GUI, a user can easily interact with the system to start investing. Before the system is implemented, it passed several tests and has proven to be stable across platforms. All the components designed so as to not hinder user experience while the credit mining system is active.

The performance of the credit mining system met our expectations. All the components were did their tasks properly. Prospecting is both fast and accurate to find and filter swarms on the Internet. The scoring policy successfully selects the most undersupplied swarms and surpasses previous policy accuracy. Moreover, it also stops both saturated and low potential swarms. With stimulating enabled, in most cases, the system use a large portion (80\%) of its resources. The upload/download ratio can reach up to 4.91 with an average of 3.71, although the target was only 2.0. After making the comparison, the current system can gain more credit than in prior work. We also showed that credit miners have a beneficial impact on the community as a whole. When the number of miners is half that of the peers, the credit miners can boost more than half of the swarms in a particular community by up to 29\%. Increasing the number of miners can increase the swarm coverage as well as the average peers' download speed.

\section{Future Work}
Currently, credit miners still see other miners as a normal peer. There might be a case in which a miner seeds to another miner, which is unnecessary. The key problem of "Co-Investors" is to recognize and utilize the existence of other miners. When recognizing other miners, investment can be more selective. There is less need to boost a swarm if there are already miners there, for example. 

Although we proposed the scoring policy in this thesis, the optimal \textit{multipliers} to reach the highest gain possible are still unknown. With many parameters, a study to find the weight and importance of those parameters is desired. Furthermore, this policy can be extended by adding more parameters while adopting the same calculation method.

Another aspect that still unknown is whether using different policies for different swarms is beneficial. Currently, the policy cannot be changed, and all of the swarms need to comply with a single policy. Moreover, as we have shown in the previous chapter, not all swarms are suitable for stimulation. The \textit{partial mining} mechanism is a method to apply different policies and optimizations to different swarm, in order to get the highest credit gain possible. 