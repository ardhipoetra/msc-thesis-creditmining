\chapter{Economics in \bt}

%As mentioned before, \bt~community can be viewed as social networking. Each peer represents a user. A user has \textit{needs}, which is to download a file, to be fulfilled. A file is provided by another user who has different need. This situation creates supply and demand as in traditional economics. This chapter will discuss the economics foundation formed in \bt~system.

%\todo[inline]{invest vs donate. I think we need to stick to just one of them.}

% incentive -> various, centralized, decentralized. Complicated or not. modifying bittorrent protocol? -> not standard 



%\subsection{Tackling free rider problem}

% cite : The Bittorrent P2P File-Sharing System: Measurements and Analysis

% may include limitation on the effectiveness meulpolder
%Require the change of the system :  \cite{2008:givetogetvod:Mol}. \cite{2010:effortincentive:rahman} \cite{2015:incentivep2pgame:kang}.
% user impossible to be altursitic, not necessary be paranoid. Fairness and system design. While may be reduce performance in spsecific case.



