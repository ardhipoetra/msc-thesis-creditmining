%De aankondiging bevat de spreker, titel, plaats, datum en tijd, samenstelling van de afstudeercommissie en een korte samenvatting (maximaal 25 regels).
\thispagestyle{empty}

\noindent \textbf{Author}\\
\begin{tabular}{l}
Ardhi Putra Pratama Hartono\\
\\
\end{tabular}\\
\noindent \textbf{Title}\\
\begin{tabular}{l}
Credits in BitTorrent: designing prospecting and investment functions\\
\\
\end{tabular}\\
\noindent \textbf{MSc presentation}\\
\begin{tabular}{l}
% <MM> DD, YYYY (like \today)
TODO GRADUATION DATE\\
\\
\end{tabular}

\vspace{1.1cm}

\noindent \textbf{Graduation Committee}\\
\begin{tabular}{ll}
%examples:
Prof. Dr. Ir. J.A. Pouwelse (supervisor) & Delft University of Technology \\
Prof. Dr. Ir. S. Hamdioui & Delft University of Technology \\
Dr. Ir. C. Hauff & Delft University of Technology \\
\end{tabular}

\begin{abstract}

Slow download speed in \bt~community is mainly caused by the existence of freerider. Credit system, as one of the widely implemented incentive mechanism, is designed to tackle this issue. However, in some cases, to gain credit efficiently is difficult. Moreover, the supply and demand misalignment in swarms can result in performance deficiency. As an answer to this issue, we introduce credit mining system, an autonomous system to download pieces from selected swarms in order to gain high upload ratio. 

Our main work is to build credit mining system. Specifically, we focus on an algorithm to invest the credit in swarms. Two stages are composed: \textit{prospecting} and \textit{mining}. In \textit{prospecting}, swarm information is extensively collected and then filtered. In \textit{mining}, swarms are sorted by its potential and then selected. We also proposed \textit{scoring policy} as a method to quantify swarms into a numerical score. Each detail of the sub-algorithm is presented and elaborated.

Finally, we implement and evaluate credit mining system in both live and controlled environment. The system is fully integrated with Tribler and able to adapt to user activity. It correctly selects undersupplied swarms. In terms of advantages, a user can gain upload/download ratio up to 4.54 by using 80\% of its resource. The majority of the swarms in the community also get its average download speed increased by up to 34\%. Based on the results, we showed that the implementation of credit mining system is beneficial for both parties, especially to cover the freeriding phenomenon.

\end{abstract}

\clearpage

