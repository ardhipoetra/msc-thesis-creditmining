%De aankondiging bevat de spreker, titel, plaats, datum en tijd, samenstelling van de afstudeercommissie en een korte samenvatting (maximaal 25 regels).
\thispagestyle{empty}

\noindent \textbf{Author}\\
\begin{tabular}{l}
Ardhi Putra Pratama Hartono\\
\\
\end{tabular}\\
\noindent \textbf{Title}\\
\begin{tabular}{l}
Credits in BitTorrent: designing prospecting and investment functions\\
\\
\end{tabular}\\
\noindent \textbf{MSc presentation}\\
\begin{tabular}{l}
% <MM> DD, YYYY (like \today)
TODO GRADUATION DATE\\
\\
\end{tabular}

\vspace{1.1cm}

\noindent \textbf{Graduation Committee}\\
\begin{tabular}{ll}
%examples:
Prof. Dr. Ir. J.A. Pouwelse (supervisor) & Delft University of Technology \\
Prof. Dr. Ir. S. Hamdioui & Delft University of Technology \\
Dr. Ir. C. Hauff & Delft University of Technology \\
\end{tabular}

\begin{abstract}

Slow download speed in the \bt~community is mainly caused by the existence of freeriders. The credit system, as one of the most widely implemented incentive mechanisms, is designed to tackle this issue. However, in some cases, gaining credit efficiently is difficult. Moreover, the supply and demand misalignment in swarms can result in performance deficiency. As an answer to this issue, we introduce a credit mining system, an autonomous system to download pieces from selected swarms in order to gain a high a upload ratio. 

Our main work is to build a credit mining system. Specifically, we focused on an algorithm to invest the credit in swarms. This is composed of two stages: \textit{prospecting} and \textit{mining}. In \textit{prospecting}, swarm information is extensively collected and then filtered. In \textit{mining}, swarms are sorted by their potential and then selected. We also propose a \textit{scoring policy} as a method to quantify swarms with a numerical score. Each detail of the sub-algorithm is presented and elaborated.

Finally, we implemented and evaluated the credit mining system in both live and controlled environments. The system is fully integrated with Tribler and is able to adapt to user activity, while correctly selecting undersupplied swarms. In terms of advantages, users can gain an upload/download ratio of up to 4.54 by using 80\% of their resources. The majority of the swarms in the community also get their average download speed increased by up to 34\%. Based on the results, we showed that the implementation of the credit mining system is beneficial for both parties, especially considering the freeriding phenomenon.

\end{abstract}

\clearpage

